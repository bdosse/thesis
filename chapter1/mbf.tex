\section{Mouvement Brownien fractionnaire}

Selon le théorème d'extension de Kolmogorov, il est possible de
définir un processus stochastique gaussien à partir de son opérateur
de covariance et de sa fonction moyenne. Nous proposons de définir de
la sorte le mouvement Brownien fractionnaire, qui jouit de propriétés
similaires au mouvement Brownien standard.

Ses applications, nombreuses et transversales, suffisent à justifier
l'étude sommaire ci-présentée. Pour de plus amples détails sur les
propriétés génériques du mouvement Brownien fractionnaire, ou plus
généralement, des processus $\alpha$-stables, on pourra consulter le
livre de référence~\cite{samorodnitsky1994}. Concernant l'intégration
stochastique définie à partir du mouvement Brownien fractionnaire,
\cite{biagini2008} propose une étude détaillée de multiples
définitions de ce type d'intégrales ; on trouvera en particulier des
applications en finance et en théorie de la commande
optimale. L'ouvrage~\cite{mishura2008} s'attache à étudier les
équations différentielles stochastiques bâties sur ce processus.

\begin{alert}
  Ajouter un rappel historique sur le mBf
  (voir~\cite[p.~6]{biagini2008} et~\cite{mandelbrot1968}).
\end{alert}
\subsection{Définition et propriétés}

Nous choisissons d'étudier le mouvement Brownien fractionnaire défini
à partir de son opérateur de covariance. L'approche présentée ici est
développée dans~\cite{ayache2018}; notons que, historiquement, c'est
par une intégrale stochastique qu'a été défini le mouvement Brownien
fractionnaire (voir~\cite{mandelbrot1968} à ce propos).

\begin{definition}\label{def:mBf}
  Soit $H\in\oset{0}{1}$; un processus stochastique
  $B_H = \{B_H(t) : t\geq 0\}$ est un \emph{mouvement Brownien
    fractionnaire}\index{mouvement Brownien fractionnaire} (noté dans
  la suite \og{}mBf\fg{}) s'il s'agit d'un processus gaussien centré
  dont l'opérateur de covariance $K_H$ est défini par
  \[ K_H(t,s) = \Esp{B_H(t)B_H(s)} = \frac{1}{2}(t^{2H} + s^{2H} -
    |t-s|^{2H}) \]
  et dont les trajectoires sont presque sûrement continues.
\end{definition}

\nomenclature{mBf}{Mouvement Brownien fractionnaire}

\begin{alert}
  Renvoyer en annexe le résultat donnant l'existence d'un processus
  gaussien pour un opérateur de covariance donné.
\end{alert}

\begin{question}
  Proposer un préfacteur $\frac{c}{2}$ (au lieu de $\frac{1}{2}$) pour
  donner un processus dont la variance vaut $ct$?
\end{question}

Il est clair que pour $H= \frac{1}{2}$, $B_H$ est un mouvement
Brownien standard.

\begin{remarque}
  Dans la~\Cref{def\string:mBf}, l'hypothèse d'un processus centré
  peut être relâchée. Cela ne change fondamentalement aucun résultat
  présenté ici. Nous la formulons afin de ne pas alourdir les
  notations et les calculs.
\end{remarque}

Le lien avec le mouvement Brownien standard peut sembler
artificiel. La proposition suivante rend ce lien plus évident à l'aide
d'une représentation intégrale\index{représentation
  intégrale!mouvement Brownien fractionnaire}.

\begin{proposition}[Représentation intégrale]
  Soit $H\in\oset{0}{1}$; le processus définit explicitement par
  \begin{gather*}
    B_H(0) = 0 \\
    B_H(t) - B_H(0) = c\left(\int_{-\infty}^0 (t-s)^{H- \frac{1}{2}} - (-s)^{H- \frac{1}{2} } dB(s) + \int_0^t (t-s)^{H- \frac{1}{2}} dB(s)\right)
  \end{gather*}
  où $c\in\oset{0}{\infty}$ est une constante, est un mBf.
\end{proposition}

\begin{alert}
  On pourra écrire quelque part un rappel portant sur l'intégrale
  d'Itô. Il faudra prendre soin de préciser que même si on ne l'a pas
  étudié dans le cours \og{}Processus stochastiques\fg{} (2021--2022),
  on peut définir une intégrale d'Itô impropre, \emph{e.g.} sur des
  intervalles non bornés.
\end{alert}

On peut d'ores et déjà donner des propriétés fondamentales du mBf.

\begin{proposition}
  \label{pro:fbm-basic-properties}
  Soient $H\in\oset{0}{1}$ et $B_H = \{B_H(t) : t\geq 0\}$ un mBf. On
  a :
  \begin{enumerate}
  \item $B_H(0) = 0$ p.s.
  \item \emph{(Accroissements stationnaires)} quels que soient $h>0$
    et $t\geq 0$, $B_H(t+h)-B_H(t)\sim B_H(h)$.
  \item \emph{(Auto-similarité)} quels que soient $\alpha>0$ et
    $t\geq 0$, $\alpha^{-H} B_H(\alpha t) \sim B_H(t)$
  \item \emph{(Inversion du temps)} quel que soit $t\geq 0$,
    $t^{2H}B_H(\frac{1}{t}) \sim B_H(t)$.
  \end{enumerate}
\end{proposition}
\begin{proof}
  \begin{alert}
    À faire.
  \end{alert}
\end{proof}

\begin{question}
  Comment apporter les distinctions portant sur les valeurs possibles
  de $H$ : évoquer tout de go que le cas $H=\frac{1}{2}$ crée
  naturellement cette distinction, ou attendre que cela émerge des
  calculs ?
\end{question}

\subsection{Propriétés et régularité}
L'exposant de Hurst modifie la régularité trajectorielle du fBm, mais
modifie également quelques propriétés en comparaison du mouvement
Brownien standard. Le mBf est auto-similaire, mais ses incréments ne
sont pas indépendants. En effet, soit $B_H$ un mBf, et considérons
$h>0$, et $t,s\geq 0$ tels que $0\leq t<t+h\leq s$ ; il vient
successivement

\begin{align*}
  \Cov((B_H(t+h)-B_H(t)),(B_H(s+h)-B_H(s))) &= \Esp{(B_H(t+h)-B_H(t))(B_H(s+h)-B_H(s))} \\
                                            &= -|t-s|^{2H}+\frac{1}{2}|t-s-h|^{2H}+\frac{1}{2}|t-s+h|^{2H}.
\end{align*}
Et en notant $|t-s|=nh$, la covariance s'écrit :
\begin{equation*}
  \Cov((B_H(t+h)-B_H(t)),(B_H(s+h)-B_H(s))) = \frac{1}{2}\left((n-1)^{2H} + (n+1)^{2H} - 2n^{2H}\right)h^{2H}
\end{equation*}

Définissons le processus
$X = \left\{X_n = B_H(n+1)-B_H(n) | t\in\N\right\}$ ; vu
la~\Cref{pro\string:fbm-basic-properties}, $(X_n)_{n\in\N}$ est un
processus stationnaire. La fonction $\rho_H:\N\to\R$ définie par
\begin{equation*}
  \rho_H(n) = \Cov(X_{k+n}, X_k)
\end{equation*}
est donc bien définie quel que soit $k\in\N$, et son comportement
dépend uniquement de la valeur de $H$. En effet, si $H=\frac{1}{2}$,
il est clair que $\rho_H$ est identiquement nulle. Si $H<\frac{1}{2}$, alors la série
\begin{equation}
  \label{eqn:rho-series}
  \sum_{n\in\N}^{} \rho_H(n)
\end{equation}
est absolument convergente, et on dira $B_H$ est un processus
\emph{dépendant à court terme} (ou \emph{à mémoire courte}). Si
$H>\frac{1}{2}$, alors la série~\ref{eqn:rho-series} diverge, et on
dira que $B_H$ est un processus \emph{dépendant à long terme} (ou
\emph{à mémoire longue}).

\begin{prerequis}
  On peut définir ce qu'est une semimartingale, mais il faut alors
  définir ce qu'est une martingale locale. Soit on suppose que c'est
  connu, soit on donne tout de même quelques propriétés de ces objets.
\end{prerequis}

On peut définir \emph{plusieurs} intégrales stochastiques pour le fBm
: voir~\cite[Chp.~II-V]{biagini2008}. C'est moins agréables qu'avec le
mouvement Brownien standard.

Tout comme pour le mouvement Brownien standard, on utilise le théorème
de Kolmogorov-Centsov pour établir la régularité des trajectoires du
fBm.

\begin{alert}
  \textsc{Biagini} énonce que le fBm possède une mémoire longue. Cette
  propriété est davantage étudiée pour les séries temporelles.
\end{alert}