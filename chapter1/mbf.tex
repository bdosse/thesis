\section{Mouvement Brownien fractionnaire}
\subsection{Définition et propriétés}

Le mouvement Brownien fractionnaire (on écrira mBf à partir de
maintenant) est \emph{une} gé\-né\-ra\-li\-sa\-tion du mouvement
Brownien. Cette gé\-né\-ra\-li\-sa\-tion permet de définir la
régularité (au sens de Hölder) des trajectoires du processus. Pour
cela, on utilise l'exposant (dit \emph{de Hurst}) $H$.

\begin{question}
  Problème de l'estimation de $H$ : a-t-il sa place dans ce document ?
\end{question}

Lévy propose d'utiliser l'intégrale de Riemann-Liouville(-Itô) pour
définir le mBf :
\begin{equation}
  B_H = \frac{1}{\Gamma(H + \frac{1}{2})} \int_0^t (t-s)^{H-\frac{1}{2}} dB(s)
\end{equation}

Or, d'après~\cite{mandelbrot1968}, \emph{\og cette intégrale accorde
  bien trop d'importance à l'origine pour beaucoup d'applications
  \fg{}}. La re\-pré\-sen\-ta\-tion de Weyl est alors proposée :
\begin{gather}
  B_H(0) = 0\\
  B_H(t) = \frac{1}{\Gamma(H+\frac{1}{2})}\left(\int_{-\infty}^0(t-s)^{H-\frac{1}{2}}-(-s)^{H-\frac{1}{2}}dB(s) + \int_0^t(t-s)^{H-\frac{1}{2}}dB(s) \right)
\end{gather}
où les égalités ont lieu p.s.

\begin{alert}
  Donner la définition du mBf \emph{via} l'espérance.
\end{alert}

\begin{alert}
  On obtient gratuitement une représentation intégrale du mBf grâce à
  la re\-pré\-sen\-ta\-tion de Weyl. Il suffira de calculer l'espérance.
\end{alert}

\subsection{Propriétés et régularité}
L'exposant de Hurst modifie la régularité trajectorielle du fBm, mais
modifie également quelques propriétés en comparaison du mouvement
Brownien standard. Le mBf est auto-similaire, mais ses incréments ne
sont pas indépendants, et il ne s'agit pas d'une semimartingale.

\begin{prerequis}
  On peut définir ce qu'est une semimartingale, mais il faut alors
  définir ce qu'est une martingale locale. Soit on suppose que c'est
  connu, soit on donne tout de même quelques propriétés de ces objets.
\end{prerequis}

On peut définir \emph{plusieurs} intégrales stochastiques pour le fBm
: voir~\cite[Chp.~II-V]{biagini2008}. C'est moins agréables qu'avec le
mouvement Brownien standard.

Tout comme pour le mouvement Brownien standard, on utilise le théorème
de Kolmogorov-Centsov pour établir la régularité des trajectoires du
fBm.

\begin{alert}
  \textsc{Biagini} énonce que le fBm possède une mémoire longue. Cette
  propriété est davantage étudiée pour les séries temporelles.
\end{alert}