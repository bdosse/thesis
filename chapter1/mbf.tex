\section{Mouvement Brownien fractionnaire}

Bien que le mouvement Brownien fractionnaire possède des applications
multiples et transversales, nous nous limiterons à une étude sommaire
dans cette section. Pour de plus amples détails sur les propriétés
génériques du mouvement Brownien fractionnaire, ou plus généralement,
des processus $\alpha$-stables, on pourra consulter le livre de
référence~\cite{samorodnitsky1994}. Concernant l'intégration
stochastique définie à partir du mouvement Brownien fractionnaire,
\cite{biagini2008} propose une étude détaillée de multiples
définitions de ce type d'intégrales ; on trouvera en particulier des
applications en finance et en théorie de la commande
optimale. L'ouvrage~\cite{mishura2008} s'attache à étudier les
équations différentielles stochastiques bâties sur ce processus.

Les résultats de cette section sont issus principalement
de~\cite{biagini2008,nourdin2012}.

\subsection{Définition et propriétés}
\label{subsec\string:fBm-def}

Par le~\Cref{thm\string:kolmogorov}, le mouvement Brownien
fractionnaire peut être défini à partir de son opérateur de
covariance.

\begin{definition}
  \label{def\string:fBm}
  Soit $H\in\oset{0}{1}$ ; un \emph{mouvement Brownien fractionnaire
    d'exposant de Hurst $H$} est un processus stochastique gaussien
  centré noté
  \[ B^H(t) = \left\{B^H(t) : t \in \coset{0}{\infty}\right\} \] tel
  que
  \begin{align}
    \label{eqn\string:fBm-cov}
    K^H(t_1,t_2) = \Esp{B^H(t_1)B^H(t_2)} = \frac{1}{2}\left(t_1^{2H}+t_2^{2H}-|t_1-t_2|^{2H}\right)
  \end{align}
  et tel que \[ B^H(0) = 0 \text{ p.s.} \] quels que soient
  $t_1,t_2\in\coset{0}{\infty}$.
\end{definition}

\begin{alert}
  Ajouter package nécessaire pour guillemets francophones.
\end{alert}

\begin{notation}
  Quand il n'y aura aucune ambiguïté, la mention \emph{\og d'exposant
    de Hurst $H$\fg{}} pourra être omise. De même, on pourra désigner
  un mouvement Brownien fractionnaire par le sigle \emph{\og
    mBf\fg{}}.
\end{notation}

Comme nous pourrons le constater sous peu, l'exposant de Hurst $H$ est
un paramètre quantifiant la régularité des trajectoires du mBf
$B^H$. En particulier, plus la valeur de $H$ est proche de $0$, plus
ses trajectoires seront \emph{\og{}irrégulières\fg{}} (dans un sens
que nous prendrons soin de préciser) ; à l'inverse, plus la valeur de
$H$ est proche de $1$, plus les trajectoires de $B^H$ seront
\emph{\og{}régulières\fg{}}.

\begin{alert}
  Insérer illustration pour différentes valeurs de $H$ (\emph{i.e.}
  $H\in\{0.001,0.25,0.5,0.75,0.999\}$).
\end{alert}

\begin{remarque}
  \label{rk\string:fBm-basic}
  L'exposant de Hurst $H=\frac{1}{2}$ crée un mouvement Brownien
  standard ; si $t_1,t_2\in\coset{0}{\infty}$, et si $t_1<t_2$, alors
  \begin{align*}
    \Esp{B^{\frac{1}{2}}(t_1)B^{\frac{1}{2}}(t_2)} &= \frac{1}{2}\left(t_1+t_2-|t_1-t_2|\right)\\
                                                   &=\frac{1}{2}\left(t_1+t_2-t_2+t_1\right)\\
                                                   &= t_1 = \min\{t_1,t_2\}.
  \end{align*}

  De même, il est possible de relâcher la définition du mBf en donnant
  la permission à $H$ de prendre la valeur $1$ : dans ce cas, le mBf
  est presque sûrement une droite. En effet, quel que soit
  $t\in\coset{0}{\infty}$
  \begin{align*}
    \Esp{\left(B^1(t)-tB^1(1)\right)^2} &= \Esp{\left(B^1(t)\right)^2}-2t\Esp{B^1(t)B^1(1)}+\Esp{\left(tB^1(1)\right)^2}\\
                                        &= t^2 -t\left(t^2+1-(t-1)^2\right) + t^2=0.
  \end{align*}
\end{remarque}

\subsubsection{Propriétés fondamentales}
\label{subsubsec\string:fBm-prop}

La~\Cref{rk\string:fBm-basic} suggère que le mBf partage des
propriétés analogues au mouvement Brownien standard. La proposition
suivante donne une caractérisation du mBf.

\begin{proposition}
  Soit $H\in\oset{0}{1}$ ; si $B^H$ est un mBf, alors pour tout
  $t\in\coset{0}{\infty}$ :
  \begin{enumerate}
  \item\label{it\string:fBm-selfsim} Quel que soit $k\in\oset{0}{\infty}$,
    $B^H(kt) = k^HB^H(t)$ ;
  \item\label{it\string:fBm-stat-of-inc} Quel que soit
    $h\in\oset{0}{\infty}$, $B^H(t+h) - B^H(h) = B^H(t)$ ;
  \item\label{it\string:fBm-time-inversion} Si $t\neq 0$,
    $B^H(\frac{1}{t}) = t^{-2H}B^H(t)$ ;
  \end{enumerate}
  où la convergence a lieu au sens des lois fini-dimensionnelles.

  Réciproquement, si
  $B^H=\left\{B^H(t) : t\in\coset{0}{\infty}\right\}$ est un processus
  gaussien vérifiant les
  \cref{it\string:fBm-selfsim,it\string:fBm-stat-of-inc}, et tel que
  \begin{enumerate}[resume]
  \item $B^H(0)=0$ ;
  \item $B^H(1)=1$ ;
  \end{enumerate}
  alors $B^H$ est un mBf d'exposant de Hurst $H$.
\end{proposition}

\begin{remarque}
  \label{rk\string:fBm-selfsim-rephrase}
  La relation~(\ref{it\string:fBm-selfsim}) permet d'écrire, pour tout
  $t\in\coset{0}{\infty}$ : \[ B^H(t) = t^HB^H(1) .\]
\end{remarque}

\begin{proof}
  Pour montrer les
  \cref{it\string:fBm-selfsim,it\string:fBm-stat-of-inc,it\string:fBm-time-inversion},
  il suffit de calculer l'opérateur de covariance dans chacun de ces
  cas, et d'appliquer le~\Cref{thm\string:kolmogorov}.

  Pour montrer la réciproque, supposons disposer d'un processus $B^H$
  satisfaisant les hypothèses de celle-ci. Il suffit de montrer que le
  processus est centré (direct vu les
  points~\cref{it\string:fBm-selfsim, it\string:fBm-stat-of-inc}) et
  que son opérateur de covariance vérifie la
  relation~\cref{eqn\string:fBm-cov}. Pour ce dernier point, soient
  $t_1,t_2\in\coset{0}{\infty}$ ; il vient :
  \begin{align*}
    \Esp{B^H(t_1)B^H(t_2)} &= \frac{1}{2}\left(\Esp{\left(B^H(t_1)\right)^2}+\Esp{\left(B^H(t_2)\right)^2}-\Esp{\left(B^H(t_1)-B^H(t_2)\right)^2}\right)\\
                           &= \frac{1}{2}\left(\Esp{\left(B^H(t_1)\right)^2}+\Esp{\left(B^H(t_2)\right)^2}-\Esp{\left(B^H(|t_1-t_2|)\right)^2}\right)
  \end{align*}
  vu l'hypothèse~\cref{it\string:fBm-stat-of-inc}. De même,
  l'hypothèse~\cref{it\string:fBm-selfsim} et
  la~\Cref{rk\string:fBm-selfsim-rephrase} permette d'écrire :
  \begin{align*}
    \Esp{B^H(t_1)B^H(t_2)} &= \frac{1}{2}\Esp{\left(B^H(1)\right)^2}\left(t_1^{2H}+t_2^{2H}-|t_1-t_2|^{2H}\right)\\
                           &= \frac{1}{2}\left(t_1^{2H}+t_2^{2H}-|t_1-t_2|^{2H}\right).
  \end{align*}
  Ce qui achève la preuve.
\end{proof}

Vu la proposition précédente, on dira que le processus $B^H$ est
\emph{auto-similaire} s'il vérifie la
relation~\cref{it\string:fBm-selfsim}, qu'il est à \emph{incréments
  stationnaires} s'il vérifie la
relation~\cref{it\string:fBm-stat-of-inc}, et qu'il respecte
\emph{l'inversion du temps} s'il vérifie la
relation~\cref{it\string:fBm-time-inversion}.

\subsection{Propriétés négatives}
\label{subsec\string:fBm-absence-of-props}

Le mouvement Brownien standard, en dépit de ses trajectoires (presque
sûrement) nulle part dérivables, reste un processus plutôt
régulier. En quelques sortes, il est \emph{\og régulier dans son
  irrégularité \fg{}}. En considération de la
\Cref{rk\string:fBm-basic}, le mouvement Brownien standard est un mBf
particulier. En fait, il est \emph{maximal} pour certaines propriétés
liées à la régularité.

\subsubsection{Prédictibilité}
\label{subsubsec\string:fBm-predictability}

Une manière de mesurer la régularité d'un processus stochastique est
d'observer sa \og prédictibilité \fg{}, \emph{i.e.} s'il est possible
de prévoir, à plus ou moins long terme, son comportement. Dans le cas
du mouvement Brownien standard, il est connu\footnote{Insérer
  référence.} qu'il s'agit d'une semi-martingale. Ce n'est pas le cas
du mBf.

\subsubsection{Propriété de Markov}
\label{subsubsec\string:fBm-markov}

Un processus est vérifie la \emph{propriété de Markov} si, pour
prédire son comportement futur, seul l'instant présent est \og utile
\fg.

\begin{definition}
  Soit $X = \left\{X(t) : t\geq 0\right\}$ un processus stochastique;
  on dira de $X$ qu'il vérifie la propriété de Markov si, quel que
  soit $t_1,t_2\geq 0$ tels que $t_1<t_2$ et quel que soit $A\in\B$,
  la relation
  \begin{align}
    P(X(t_2) \in A| X(t), t\leq t_1) = P(X(t_2) \in A| X(t_1))
  \end{align}
  est vérifiée; on dira aussi que $X$ est un \emph{processus de
    Markov}.
\end{definition}

Afin de montrer que si $B^H$ est un mBf avec $H\neq\frac{1}{2}$, alors
$B^H$ ne vérifie pas la propriété de Markov, nous avons besoin du
résultat suivant.

\begin{lemme}
  Si $X = \{X(t) : t\geq 0\}$ est un processus de Markov, alors $X$
  vérifie la relation
  \begin{align}
    \Esp{X(t)X(v)}\Esp{(X(u)^2} = \Esp{X(t)X(u)}\Esp{X(u)X(v)}
  \end{align}
  quels que soient $t,u,v\in\coset{0}{\infty}$ tels que
  $t\leq u\leq v$.
\end{lemme}

\begin{theoreme}
  Soit $H\in\oset{0}{1}\setminus\frac{1}{2}$; si $B^H$ est un mBf,
  alors $B^H$ n'est pas un processus de Markov.
\end{theoreme}

\subsection{Écritures alternatives}
\label{subsec\string:fBm-other-forms}

