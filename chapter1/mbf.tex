\section{Mouvement Brownien fractionnaire}
\begin{alert}
  Ajouter un rappel historique sur le mBf
  (voir~\cite[p.~6]{biagini2008} et~\cite{mandelbrot1968}).
\end{alert}
\subsection{Définition et propriétés}

Le mouvement Brownien fractionnaire (on écrira mBf à partir de
maintenant) est \emph{une} gé\-né\-ra\-li\-sa\-tion du mouvement
Brownien. Cette gé\-né\-ra\-li\-sa\-tion permet de définir la
régularité (au sens de Hölder) des trajectoires du processus. Pour
cela, on utilise l'exposant (dit \emph{de Hurst}) $H$.

\begin{question}
  Problème de l'estimation de $H$ : a-t-il sa place dans ce document ?
\end{question}

Lévy propose d'utiliser l'intégrale de Riemann-Liouville(-Itô) pour
définir le mBf :
\begin{equation}
  B_H = \frac{1}{\Gamma(H + \frac{1}{2})} \int_0^t (t-s)^{H-\frac{1}{2}} dB(s)
\end{equation}

Or, d'après~\cite{mandelbrot1968}, \emph{\og cette intégrale accorde
  bien trop d'importance à l'origine pour beaucoup
  d'ap\-pli\-ca\-tions \fg{}}. La re\-pré\-sen\-ta\-tion de Weyl est
alors proposée :
\begin{gather}
  B_H(0) = 0\\
  B_H(t) = \frac{1}{\Gamma(H+\frac{1}{2})}\left(\int_{-\infty}^0(t-s)^{H-\frac{1}{2}}-(-s)^{H-\frac{1}{2}}dB(s) + \int_0^t(t-s)^{H-\frac{1}{2}}dB(s) \right)
\end{gather}
où les égalités ont lieu p.s.

\begin{alert}
  On obtient gratuitement une représentation intégrale du mBf grâce à
  la re\-pré\-sen\-ta\-tion de Weyl. Il suffira de calculer l'espérance.
\end{alert}

Nous proposons de définir le mouvement Brownien fractionnaire au moyen
d'une fonction de covariance, puis de revenir à cette forme intégrale
par la suite.

\begin{definition}
  Soit $H \in \oset{0}{1}$ ; un \emph{mouvement Brownien
    fractionnaire} (on écrira mBf par la suite) \emph{d'exposant de
    Hurst} $H$ est un processus gaussien centré
  $B_H = \{B_H(t) : t \geq 0\}$ dont l'opérateur de covariance
  $\corr(t,t')$ est donné par
  \begin{equation}
    \corr_{B_H}(t,t') = \Esp{B_H(t)B_H(t')} = \frac{1}{2}(t^{2H}-s^{2H}-|t-s|^{2H})
  \end{equation}
  et dont les trajectoires sont presque sûrement continues.
\end{definition}

\begin{alert}
  Renvoyer en annexe le résultat donnant l'existence d'un processus
  gaussien pour un opérateur de covariance donné.
\end{alert}

\begin{remarque}
  Si $H=\frac{1}{2}$, on obtient un mouvement Brownien standard.
  \begin{alert}
    Faire le calcul.
  \end{alert}
\end{remarque}

On peut d'ores et déjà donner des propriétés fondamentales du mBf.

\begin{proposition}
  Soient $H\in\oset{0}{1}$ et $B_H = \{B_H(t) : t\geq 0\}$ un fBm. On
  a :
  \begin{enumerate}
  \item $B_H(0) = 0$ p.s.
  \item $B_H$ est à accroissements stationnaires
  \item $B_H$ est un processus gaussien tel que $\Esp{B_H(t)^2} = t^{2H}$
  \end{enumerate}
\end{proposition}
\begin{alert}
  Démonstration à faire.
\end{alert}

\begin{question}
  Comment apporter les distinctions portant sur les valeurs possibles
  de $H$ : évoquer tout de go que le cas $H=\frac{1}{2}$ crée
  naturellement cette distinction, ou attendre que cela émerge des
  calculs ?
\end{question}

\subsection{Propriétés et régularité}
L'exposant de Hurst modifie la régularité trajectorielle du fBm, mais
modifie également quelques propriétés en comparaison du mouvement
Brownien standard. Le mBf est auto-similaire, mais ses incréments ne
sont pas indépendants, et il ne s'agit pas d'une semimartingale.

\begin{prerequis}
  On peut définir ce qu'est une semimartingale, mais il faut alors
  définir ce qu'est une martingale locale. Soit on suppose que c'est
  connu, soit on donne tout de même quelques propriétés de ces objets.
\end{prerequis}

On peut définir \emph{plusieurs} intégrales stochastiques pour le fBm
: voir~\cite[Chp.~II-V]{biagini2008}. C'est moins agréables qu'avec le
mouvement Brownien standard.

Tout comme pour le mouvement Brownien standard, on utilise le théorème
de Kolmogorov-Centsov pour établir la régularité des trajectoires du
fBm.

\begin{alert}
  \textsc{Biagini} énonce que le fBm possède une mémoire longue. Cette
  propriété est davantage étudiée pour les séries temporelles.
\end{alert}