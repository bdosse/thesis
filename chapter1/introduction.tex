On peut définir le mouvement Brownien (standard) au moyen de sa
représentation intégrale. On peut également le définir au moyen de sa
fonction moyenne $\mu$ et son opérateur de covariance $K$. Plus
précisément, si $B = \{B(t) : t \in \coset{0}{\infty}\}$ désigne un
mouvement Brownien standard sur l'espace de probabilité filtré
$\left(\Omega, \sA, \sF, \nP\right)$, alors la fonction moyenne $\mu$
est donnée par $\mu(t) = 0$ et l'opérateur de covariance est donnée
par $K(t,s) = \min(t,s)$.

\begin{alert}
  Insérer entre une et trois réalisations du mouvement Brownien sur un
  intervalle $\cset{0}{T}$.
\end{alert}

Ce chapitre introduit le \emph{\og mouvement Brownien
  fractionnaire\fg{}}, où la régularité des trajectoires est contrôlée
par une constante $H$ appelée exposant de Hurst. Après l'étude de
quelques propriétés, le \emph{\og mouvement Brownien
  multifractionnaire\fg{}} est introduit. L'évolution la plus notable
entre ces deux processus est le choix de l'exposant de Hurst : alors
qu'il est constant pour le premier, on considère une fonction qui
dépend du temps pour le second ; une conséquence immédiate est
l'évolution, au long des trajectoires, de la régularité de celles-ci.
