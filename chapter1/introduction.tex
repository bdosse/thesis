On peut définir le mouvement Brownien (standard) au moyen de sa
représentation intégrale. On peut également le définir au moyen de sa
fonction moyenne $\mu$ et son opérateur de covariance $K$. Plus
précisément, si $B = \{B(t) : t \in \coset{0}{\infty}\}$ désigne un
mouvement Brownien standard sur l'espace de probabilité filtré
$\left(\Omega, \sA, \sF, \nP\right)$, alors la fonction moyenne $\mu$
est donnée par $\mu(t) = 0$ et la fonction de covariance est donnée
par $K(t,s) = \min(t,s)$.

Ce chapitre introduit le mouvement Brownien fractionnaire, où la
régularité est contrôlée par une constante $H$ appelée exposant de
Hurst. Après l'étude de quelques propriétés, notamment en ce qui
concerne la régularité (sous différentes acceptions) des trajectoires,
le mouvement Brownien multifractionnaire est introduit. L'évolution la
plus notable entre ces deux processus est le choix de l'exposant de
Hurst : alors qu'il est constant pour le premier, on considère une
fonction qui dépend du temps pour le second. Une conséquence immédiate
est l'évolution, au long des trajectoires, de la régularité de
celles-ci.