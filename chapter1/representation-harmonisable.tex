\section{Représentation harmonisable}

\begin{presentation}
  La notion de processus harmonisable est présentée par
  M. \textsc{Loève}~(\cite{loeve1978}). Dans ce chapitre, on notera
  $\corr_X(t_1,t_2)$ l'opérateur de covariance d'un processus $X$ aux
  temps $t_1, t_2$, pour autant que $\Esp{|X|^2} < \infty$. Sauf
  mention contraire, on supposera que $X$ est centré. S'il existe un
  opérateur de covariance $\gamma$ de variation bornée telle
  que\footnote{On observe clairement une composante d'un produit
    vectoriel, ou bien un déterminant. Essayer d'investiguer dans la
    direction de ces notations.}
  \[ \corr_X(t_1,t_2) = \iint e^{i(t_1s_1 - t_2s_2)} d\gamma(s_1,s_2)
    ,\] alors on dira que $\corr_X$ est \emph{harmonisable}.
\end{presentation}

\begin{remarque}
  Sans que cela n'occasionne d'ambiguïtés, nous noterons
  $X\in L^2(\Omega)$ (voire $X\in L^2$ si le contexte est clair) pour
  exprimer que $\Esp{|X(t)|^2} < \infty$ quel que soit $t\in I$.
\end{remarque}

\section{Intégrale stochastique}

\begin{prerequis}
  Il faut introduire l'intégrale de Riemann-Stieltjes pour les
  processus (\cite[p.~138]{loeve1978}) pour donner un sens aux
  intégrales.
\end{prerequis}

\begin{alert}
  Insérer une référence vers la construction de l'intégrale d'Itô.
\end{alert}

L'intégrale d'Itô est une intégrale \emph{\og à la Riemann \fg{}} pour
une classe de processus stochastiques. Cette définition de l'intégrale
stochastique demande que les processus à intégrer soient, entre autres
hypothèses, \emph{adaptés} à la filtration du mouvement Brownien.

\begin{alert}
  Il semblerait que \textsc{Loève} ne veut pas donner trop de détails
  concernant la construction de l'intégrale : il s'agit d'une somme de
  Riemann-Stieltjes, et si elle converge dans $L^2$, alors on dit que
  c'est une intégrale ; sinon, on oublie.
\end{alert}

\begin{question}
  Refaire le raisonnement avec les processus de la classe $\Cal{H}^2$
  adaptés au processus par rapport auquel on intègre ?
\end{question}

\nomenclature{$\corr_X$}{Opérateur de covariance d'un processus
  $X \in L^2(\Omega)$}

On dira qu'un processus $X\in L^2(\Omega)$ est un \emph{processus
  harmonisable} s'il existe une variable aléatoire $Y\in L^2(\Omega)$
telle que $\corr_Y$ est harmonisable et telle que
\begin{equation}
  \label{eqn:rep-harm-proc}
  X(t) = \int e^{its} dY(s)
\end{equation}
où l'égalité est entendue au sens de la convergence presque sûre. Nous
allons montrer qu'un processus est harmonisable si, et seulement si,
sa fonction d'auto-corrélation est harmonisable. Nous donnerons
ensuite des \emph{représentations harmonisables} de tels processus,
tout en donnant quelques propriétés remarquables de ceux-ci.

\begin{alert}
  On obtient une \og transformée de \emph{Fourier-Stieltjes} \fg{}. On
  peut trouver une sorte d'inverse (convergence en moyenne
  quadratique).
\end{alert}

\subsection{Représentation en série orthogonale}

\begin{alert}
  Ajout d'un post-it (\cite[p.~143]{loeve1978}).
\end{alert}

\textsc{Loève} présente deux décompositions : un développement à la
Taylor et une décomposition par des fonctions dites
\emph{propres}\footnote{Utilisation du théorème de Mercer, à renvoyer
  en annexe ?}. Après un développement peu amusant, on définit ce
qu'est un processus de second ordre stationnaire. En particulier un
processus de second ordre\footnote{C'est à dire, qui admet des moments
  d'ordre deux finis.} qui est de plus stationnaire est un processus de
second ordre stationnaire.

On sait que toute fonction $f\in L^2(\R)$ admet une décomposition en
ondelettes de la forme
\begin{equation*}
  f = \sum_{j,k\in\Z} c_{j,k}\psi_{j,k}
\end{equation*}
où la famille
${(\psi_{j,k})}_{j,k\in\Z} = \{2^{-\frac{j}{2}}\psi(2^j\cdot-k)\}$,
définie à partir d'une fonction $\psi\in L^2(\R)$, forme une base
orthonormée de l'espace $L^2(\R)$. On peut supposer qu'une telle
décomposition existe également dans le cas des processus
stochastiques, puisque ces processus peuvent être vus comme des
fonctions à deux variables. Si une telle décomposition existe bien, il
est nécessaire d'invoquer un outil d'analyse fonctionnel pour le
démontrer. Si nous l'énonçons ici, nous renvoyons le lecteur à la
Section~\ref{sec:mercers-theorem} pour en connaître une démonstration.

\begin{theorem}[Théorème de Mercer]
  \label{thm:mercers-theorem}
  Soit $I$ un intervalle fermé de $\R$ et soit
  \[ K : I\times I\to \R \] une fonction continue et symétrique. Si
  $K$ est une fonction de type positif, alors il existe une base
  orthonormée de fonctions continues $(e_k)_{k\in\N}$ de $L^2(I)$ et
  des constantes $(c_k)_{k\in\N}$ positives ou nulles telles que
  \begin{equation}
    K(s,t) = \sum_{k\in\N} c_k e_k(s) e_k(t)
  \end{equation}
  où la convergence de la série est uniforme.
\end{theorem}

\begin{remarque}
  Les fonctions $(e_j)_{j\in\N}$ intervenant dans le
  \Cref{thm:mercers-theorem} sont des \og fonctions propres\fg{}
  associées aux \og valeurs propres\fg{} $(c_k)_{k\in\N}$ pour
  l'opérateur dont le noyau est défini par $K$\footnote{On pourra par
    ailleurs trouver l'appellation de \emph{noyau de Mercer}.},
  \emph{i.e.}
  \begin{equation*}
    \int K(s,t) e_k(t) dt = c_k e_k(t).
  \end{equation*}
\end{remarque}

% Théorème de décomposition :
Le théorème suivant assure, sous certaines conditions, l'existence
d'une \og décomposition en ondelettes\fg{} d'un processus stochastique
$X\in L^2$.

\begin{theoreme}
  \label{thm:orthogonal-decomp}
  Soit $X = \{X(t) : t\in\cset{0}{T}\}$ un processus de $L^2(\Omega)$,
  et supposons que $\corr_X$ soit continue sur
  $\cset{0}{T}\times\cset{0}{T}$ ; le processus $X$ admet une
  décomposition de la forme~\ref{eqn:orthogonal-decomp} si, et
  seulement si, la suite de constante $(c_k)_{k\in\N}$ et la suite de
  fonctions $(e_k)_{k\in\N}$ sont respectivement les valeurs propres
  et les fonctions propres de $\corr_X$.
\end{theoreme}
\begin{alert}
  \begin{proof}
    À faire.
  \end{proof}
\end{alert}

\begin{remarque}
  La convergence de la série du~\Cref{thm:orthogonal-decomp} est uniforme.
\end{remarque}
\begin{alert}
  \begin{proof}
    À faire.
  \end{proof}
\end{alert}

\subsection{Représentation intégrale}
On donne ici une condition nécessaire et suffisante pour obtenir une
représentation de la forme~(\ref{eqn:rep-harm-proc}) : être
stationnaire et continu en un point.

\begin{alert}
  Les liens ne sont pas très clairs entre les différentes
  décompositions, ni comment elles peuvent être appliquées afin
  d'extraire de l'information sur les processus.
\end{alert}

\begin{question}
  Que faire des résultats concernant la stabilité (dans $L^2$)
  opérations de différentiation et d'intégration ?
\end{question}
