\section{Propriétés en moyenne quadratique}

Soit $\left(\Omega, \Cal{A}, \P\right)$ un espace probabilisé ; on
dira d'une variable aléatoire $X$ qu'elle admet un moment d'ordre deux
si, et seulement si, $\Var(X) = \Esp{X^2}-\Esp{X}^2 <\infty$, et on
note alors $X\in L^2(\Omega, \cA, \P)$, voire $X\in L^2(\Omega, \P)$
ou $X\in L^2(\P)$ si le contexte est clair. Nous allons étendre cette
définition aux processus stochastiques, et donner quelques propriétés
liées à l'existence de ces moments. Dans la suite, nous
suivons~\cite{loeve1978} ainsi que~\cite{samorodnitsky1994}.


\subsection{Représentation en série orthogonale}

\begin{alert}
  Ajout d'un post-it (\cite[p.~143]{loeve1978}).
\end{alert}

\textsc{Loève} présente deux décompositions : un développement à la
Taylor et une décomposition par des fonctions dites
\emph{propres}\footnote{Utilisation du théorème de Mercer, à renvoyer
  en annexe ?}. Après un développement peu amusant, on définit ce
qu'est un processus de second ordre stationnaire. En particulier un
processus de second ordre\footnote{C'est à dire, qui admet des moments
  d'ordre deux finis.} qui est de plus stationnaire est un processus de
second ordre stationnaire.

On sait que toute fonction $f\in L^2(\R)$ admet une décomposition en
ondelettes de la forme
\begin{equation*}
  f = \sum_{j,k\in\Z} c_{j,k}\psi_{j,k}
\end{equation*}
où la famille
${(\psi_{j,k})}_{j,k\in\Z} = \{2^{-\frac{j}{2}}\psi(2^j\cdot-k)\}$,
définie à partir d'une fonction $\psi\in L^2(\R)$, forme une base
orthonormée de l'espace $L^2(\R)$. On peut supposer qu'une telle
décomposition existe également dans le cas des processus
stochastiques, puisque ces processus peuvent être vus comme des
fonctions à deux variables. Si une telle décomposition existe bien, il
est nécessaire d'invoquer un outil d'analyse fonctionnel pour le
démontrer. Si nous l'énonçons ici, nous renvoyons le lecteur à la
\Cref{subsec\string:mercers-theorem} pour en connaître une
démonstration.

\begin{theoreme}[Théorème de Mercer]
  \label{thm:mercers-theorem}
  Soit $I$ un intervalle fermé de $\R$ et soit
  \[ K : I\times I\to \R \] une fonction continue et symétrique. Si
  $K$ est une fonction de type positif, alors il existe une base
  orthonormée de fonctions continues $(e_k)_{k\in\N}$ de $L^2(I)$ et
  des constantes $(c_k)_{k\in\N}$ positives ou nulles telles que
  \begin{equation}
    K(s,t) = \sum_{k\in\N} c_k e_k(s) e_k(t)
  \end{equation}
  où la convergence de la série est uniforme.
\end{theoreme}

\begin{remarque}
  Les fonctions $(e_j)_{j\in\N}$ intervenant dans le
  \Cref{thm\string:mercers-theorem} sont des \og fonctions propres\fg{}
  associées aux \og valeurs propres\fg{} $(c_k)_{k\in\N}$ pour
  l'opérateur dont le noyau est défini par $K$\footnote{On pourra par
    ailleurs trouver l'appellation de \emph{noyau de Mercer}.},
  \emph{i.e.}
  \begin{equation*}
    \int K(s,t) e_k(t) dt = c_k e_k(t).
  \end{equation*}
\end{remarque}

% Théorème de décomposition :
Le théorème suivant assure, sous certaines conditions, l'existence
d'une \og décomposition en ondelettes\fg{} d'un processus stochastique
$X\in L^2$.

\begin{theoreme}
  \label{thm:orthogonal-decomp}
  Soit $X = \{X(t) : t\in\cset{0}{T}\}$ un processus de $L^2(\Omega)$,
  et supposons que $\corr_X$ soit continue sur
  $\cset{0}{T}\times\cset{0}{T}$ ; le processus $X$ admet une
  décomposition de la forme~\ref{eqn:orthogonal-decomp} si, et
  seulement si, la suite de constante $(c_k)_{k\in\N}$ et la suite de
  fonctions $(e_k)_{k\in\N}$ sont respectivement les valeurs propres
  et les fonctions propres de $\corr_X$.
\end{theoreme}
\begin{alert}
  \begin{proof}
    À faire.
  \end{proof}
\end{alert}

\begin{remarque}
  La convergence de la série du~\Cref{thm\string:orthogonal-decomp} est uniforme.
\end{remarque}
\begin{alert}
  \begin{proof}
    À faire.
  \end{proof}
\end{alert}

