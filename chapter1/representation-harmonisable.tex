\section{Représentation harmonisable}
La notion de processus harmonisable est présentée par
M. \textsc{Loève}~(\cite{loeve1978}). Dans ce chapitre, on notera
$\corr_X(t_1,t_2)$ l'auto-corrélation d'un processus centré $X$ aux
temps $t_1, t_2$, pour autant que $\Esp{|X|^2} < \infty$. S'il existe
une fonction d'auto-corrélation $\gamma$ de variation bornée telle
que\footnote{On observe clairement une composante d'un produit
  vectoriel, ou bien un déterminant. Essayer d'investiguer dans la
  direction de ces notations.}
\[ \corr_X(t_1,t_2) = \iint e^{i(t_1s_1 - t_2s_2)} d\gamma(s_1,s_2)
  ,\] alors on dira que $\corr_X$ est \emph{harmonisable}.

\section{Intégrale stochastique}

\begin{prerequis}
  Il faut introduire l'intégrale de Riemann-Stieltjes pour les
  processus (\cite[p.~138]{loeve1978}) pour donner un sens aux
  intégrales.
\end{prerequis}

\begin{alert}
  Insérer une référence vers la construction de l'intégrale d'Itô.
\end{alert}

L'intégrale d'Itô est une intégrale \emph{\og à la Riemann \fg{}} pour
une classe de processus stochastiques. Cette définition de l'intégrale
stochastique demande que les processus à intégrer soient, entre autres
hypothèses, \emph{adaptés} à la filtration du mouvement Brownien.

\begin{alert}
  Il semblerait que \textsc{Loève} ne veut pas donner trop de détails
  concernant la construction de l'intégrale : il s'agit d'une somme de
  Riemann-Stieltjes, et si elle converge dans $L^2$, alors on dit que
  c'est une intégrale ; sinon, on oublie.
\end{alert}

\begin{question}
  Refaire le raisonnement avec les processus de la classe $\Cal{H}^2$
  adaptés au processus par rapport auquel on intègre ?
\end{question}

\nomenclature{$\corr_X$}{Fonction d'auto-corrélation d'un processus
  $X \in L^2(\Omega)$}

On dira qu'un processus $X\in L^2(\Omega)$ est un \emph{processus
  harmonisable} s'il existe une variable aléatoire $Y\in L^2(\Omega)$
telle que $\corr_Y$ est harmonisable et telle que
\begin{equation}
  \label{eqn:rep-harm-proc}
  X(t) = \int e^{its} dY(s)
\end{equation}
où l'égalité est entendue au sens de la convergence presque sûre. Nous
allons montrer qu'un processus est harmonisable si, et seulement si,
sa fonction d'auto-corrélation est harmonisable. Nous donnerons
ensuite des \emph{représentations harmonisables} de tels processus,
tout en donnant quelques propriétés remarquables de ceux-ci.

\begin{alert}
  On obtient une \og transformée de \emph{Fourier-Stieltjes} \fg{}. On
  peut trouver une sorte d'inverse (convergence en moyenne
  quadratique).
\end{alert}

\subsection{Représentation en série orthogonale}

\begin{alert}
  Ajout d'un post-it (\cite[p.~143]{loeve1978}).
\end{alert}

\textsc{Loève} présente deux décompositions : un développement à la
Taylor et une décomposition par des fonctions dites
\emph{propres}\footnote{Utilisation du théorème de Mercer, à renvoyer
  en annexe ?}. Après un développement peu amusant, on définit ce
qu'est un processus de second ordre stationnaire. En particulier un
processus de second ordre\footnote{C'est à dire, qui admet des moments
  d'ordre deux finis.} qui est de plus stationnaire est un processus de
second ordre stationnaire.

\subsection{Représentation intégrale}
On donne ici une condition nécessaire et suffisante pour obtenir une
représentation de la forme~(\ref{eqn:rep-harm-proc}) : être
stationnaire et continu en un point.

\begin{alert}
  Les liens ne sont pas très clairs entre les différentes
  décompositions, ni comment elles peuvent être appliquées afin
  d'extraire de l'information sur les processus.
\end{alert}

\begin{question}
  Que faire des résultats concernant la stabilité (dans $L^2$)
  opérations de différentiation et d'intégration ?
\end{question}