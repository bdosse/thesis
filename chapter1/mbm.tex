\section{Mouvement Brownien multifractionnaire}
Un inconvénient considérable du mBf est son incapacité à faire varier
la régularité des trajectoires au cours du temps. Le mouvement
Brownien multifractionnaire (noté mBm dans la suite) permet de se
passer de cette limitation en utilisant une fonction exposant de Hurst
$H(t)$.

Cette section est basée sur les résultats donnés
dans~\cite{ayache2018}, néanmoins ceux-ci sont limités à une dimension
afin de ne pas alourdir l'exposé.

\subsection{Définition et propriétés}
\label{subsec:mBm-def}

Dans la \Cref{sec:fBm-1d}, la généralisation proposée par le mBf
consistait à introduire un exposant de Hurst $H\in\oset{0}{1}$ afin
d'assujettir les trajectoires du processus à une certaine
irrégularité. Comme rappelé précédemment, l'introduction du mBf par
Mandelbrot \& Van Ness~(\cite{mandelbrot1968}) se fut par une
intégrale stochastique. Le choix d'une généralisation du mBf peut se
faire à l'aide de ces représentations : dans~\cite{peltier1995}, les
auteurs proposent d'utiliser la représentation intégrale ;
dans~\cite{ayache2018}, il est proposé de modifier la transformée de
Fourier du mBf. Dans les deux cas, cette modification porte sur
l'exposant de Hurst, remplacé par une fonction mesurable
$t\mapsto H(t)$\footnote{Dans~\cite{ayache2018}, cette fonction est
  nommée \emph{\og fonction multifractionnaire de Hurst \fg{}}.}.

\subsubsection{Propriétés du générateur}

\subsection{Propriétés des trajectoires du mBm}

\begin{prerequis}
  Pour évoquer le caractère non différentiable des trajectoires, on
  peut introduire quelques éléments relatifs aux temps locaux.
\end{prerequis}

\subsubsection{Propriété de Hölder}

\subsection{Écritures alternatives}

\subsubsection*{Intégrale stochastique}

\subsubsection*{Décomposition en ondelettes}

\begin{alert}
  Concernant la représentation en ondelettes, prendre garde à
  certaines subtilités (\cite{ayache2010}).
\end{alert}

\subsection{Un mot sur le mBm lorsque $H(t)$ n'est pas déterministe}

\begin{piste}
  On peut définir (au moins) une intégrale stochastique pour le
  mBm. C'est assez difficile~(\cite{lebovits2014}), dans le sens où il
  faut utiliser des outils qui (je crois) n'ont pas encore été vu ou
  ne sont pas au programme actuellement. On peut
  explorer~\cite{herbin2012}.
\end{piste}

\begin{piste}
  Considérer une fonction aléatoire suffisamment régulière pour la
  fonction $H(t)$. On peut procéder comme suit :
  \begin{enumerate}
  \item Remplacer $H(t)$ par un processus connus (comme $B(t)$) pour
    obtenir une première heuristique.
  \item Étudier les conséquences d'un tel remplacement (en général) en
    partant de la représentation intégrale (en particulier, est-ce
    qu'on peut \emph{toujours} calculer ladite intégrale ?)
  \item Étudier les conséquences d'un tel remplacement (en général) en
    partant de la représentation en série.
  \item Si on veut passer d'une représentation à l'autre, est-ce qu'il
    y a des difficultés supplémentaires à surmonter ?
  \item Peut-on toujours simuler les trajectoires d'un tel processus ?
  \end{enumerate}
\end{piste}
