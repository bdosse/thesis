\section{Mouvement Brownien multifractionnaire}
Un inconvénient considérable du fBm est son incapacité à faire varier
la régularité des trajectoires au cours du temps. Le mouvement
Brownien multifractionnaire (noté mBm dans la suite) permet de se
passer de cette limitation en utilisant une fonction exposant de Hurst
$H(t)$.

\subsection{Définition et propriétés}
\textsc{Ayache}~(\cite{ayache2018}) introduit le mBm au moyen d'une
représentation intégrale et de générateurs.

\begin{prerequis}
  Inclure le passage sur la transformation de Fourier : elle est utile
  pour quelques articles.
\end{prerequis}

Il existe des liens évidents avec le mBf : il suffit de choisir une
fonction constante.

\begin{question}
  Que peut-on dire de l'auto-similarité du mBm ?
\end{question}

\begin{prerequis}
  On peut définir (au moins) une intégrale stochastique pour le
  mBm. C'est assez difficile, dans le sens où il faut utiliser des
  outils qui (je crois) n'ont pas encore été vu ou ne sont pas au
  programme actuellement.
\end{prerequis}

\subsection{Régularité}
Sont données quelques propriétés trajectorielles, dont des conditions
nécessaires et suffisantes sur la fonction $H$ pour la continuité du
mBm. Des résultats concernant la régularité (globale, locale, et
ponctuelle) de Hölder sont démontrés.