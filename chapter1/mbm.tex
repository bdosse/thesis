\section{Mouvement Brownien multifractionnaire}
Un inconvénient considérable du fBm est son incapacité à faire varier
la régularité des trajectoires au cours du temps. Le mouvement
Brownien multifractionnaire (noté mBm dans la suite) permet de se
passer de cette limitation en utilisant une fonction exposant de Hurst
$H(t)$.

\subsection{Définition et propriétés}
\textsc{Ayache}~(\cite{ayache2018}) introduit le mBm au moyen d'une
représentation intégrale et de générateurs.

\begin{prerequis}
  Inclure le passage sur la transformation de Fourier : elle est utile
  pour quelques articles.
\end{prerequis}

Un générateur est une intégrale stochastique (bien choisie) par
rapport au bruit gaussien (transformation de Fourier d'un mouvement
Brownien).

Si on note $X(u,v)$ un générateur, alors $Y(t) = X(t, H(t))$ est un
mBm (cette construction considère déjà $t\in\R^d$).

Il existe des liens évidents avec le mBf : il suffit de choisir une
fonction constante.

\begin{alert}
  Écrire le théorème d'existence de l'événement de probabilité $1$
  dont il est question dans la Proposition suivante.
\end{alert}

\begin{proposition}
  Supposons que $\Omega^*$ bien choisie est de mesure $1$ ; la
  continuité de $H$ implique la continuité de $Y(\cdot,\omega)$ sur
  $\R^d$ pour tout $\omega\in\Omega^*$.
  \begin{alert}
    On a une réciproque.
  \end{alert}
\end{proposition}

\begin{question}
  Que peut-on dire de l'auto-similarité du mBm ?
  Voir~\cite[Définitions~(1.69) et~(1.70)]{ayache2018}. Le
  Théorème~1.91 de~\cite{ayache2018} répond à la question.
\end{question}

\begin{piste}
  On peut définir (au moins) une intégrale stochastique pour le
  mBm. C'est assez difficile~(\cite{lebovits2014}), dans le sens où il
  faut utiliser des outils qui (je crois) n'ont pas encore été vu ou
  ne sont pas au programme actuellement. On peut
  explorer~\cite{herbin2012}.
\end{piste}

\begin{piste}
  Considérer une fonction aléatoire suffisamment régulière pour la
  fonction $H(t)$. On peut procéder comme suit :
  \begin{enumerate}
  \item Remplacer $H(t)$ par un processus connus (comme $B(t)$) pour
    obtenir une première heuristique.
  \item Étudier les conséquences d'un tel remplacement (en général) en
    partant de la représentation intégrale (en particulier, est-ce
    qu'on peut \emph{toujours} calculer ladite intégrale ?)
  \item Étudier les conséquences d'un tel remplacement (en général) en
    partant de la représentation en série.
  \item Si on veut passer d'une représentation à l'autre, est-ce qu'il
    y a des difficultés supplémentaires à surmonter ?
  \item Peut-on toujours simuler les trajectoires d'un tel processus ?
  \end{enumerate}
\end{piste}

\subsection{Régularité}

\begin{question}
  Est-il nécessaire de rappeler les notions vues lors des projets des
  cours d'Espaces Fonctionnels ? Renvois en annexe ?
\end{question}

Sont données quelques propriétés trajectorielles, dont des conditions
nécessaires et suffisantes sur la fonction $H$ pour la continuité du
mBm. Des résultats concernant la régularité (globale, locale, et
ponctuelle) de Hölder sont démontrés.

\begin{prerequis}
  Temps local ; on montre qu'un temps local suffisamment régulier
  implique que, presque sûrement, les trajectoires du processus ne
  sont pas dérivables (voir \emph{\og le principe de Bernam
    \fg{}}~\cite[Sec.~2.3]{ayache2018}).
\end{prerequis}