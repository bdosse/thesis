On dispose de plusieurs méthodes permettant d'étendre le mouvement
Brownien fractionnaire dans l'espace euclidien $\R^d$ ; on introduit
alors les notions d'extensions isotrope\footnote{On utilise les
  matrices orthogonales pour introduire l'isotropie ; ça revient à
  demander l'invariance par une isométrie~(\cite{ayache2018}).}
(processus de Lévy) et non-isotrope (drap brownien). Ces techniques
sont reprises et étudiées dans le cas d'un mouvement Brownien
multifractionnaire.

\begin{question}
  Ce qui suit évoque des généralisations à $\R^d$ des mBf et mBm ; ne
  devrait-on pas également au moins évoquer le cas du mB standard
  (même si $B = B_H$ avec $H=\frac{1}{2}$) ?
\end{question}