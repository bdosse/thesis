\section{Cas du mBf}
\subsection{Mouvement Brownien fractionnaire de Lévy}
Il s'agit de l'extension isotrope, \emph{i.e.} là où apparaît une
valeur absolue pour le mBf, on fait apparaître une norme sur $\R^d$ ;
dès lors, l'exposant de Hurst est identique quelle que soit la
direction considérée.

\subsection{Drap Brownien fractionnaire}
La norme du mBf de Lévy est remplacée par un produit tensoriel des
covariances unidimensionnelles ; l'exposant de Hurst peut être
différent selon la direction observée.