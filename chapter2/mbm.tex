\section{Cas du mBm}
Les paramètres vivent dans $\R^d$, donc il faut définir une fonction
de Hurst $H:\R^d \to \oset{0}{1}$. Les représentations sont intégrales
: on impose de plus que $H$ soit mesurable.
\subsection{Extension isotrope}
Les constructions sont similaires au mBf.
\subsection{Extension anisotrope}
Les constructions sont similaires au mBf.
\subsection{Régularité des extensions}
La régularité dépend évidemment de la régularité de la fonction de
Hurst $H$ ; \textsc{Herbin}~(\cite{herbin2002}) exhibe une propriété
intéressante (dans le cas isotrope) : si $H \in C^\beta(\R^d)$, alors
on peut contrôler la variance des accroissements en fonction de
$\max{\beta, H(t)}$. On a une propriété similaire dans le cas
anisotrope.

\subsection{Du fBm au mBm}
\vspace{0.2em}
\begin{alert}
  À chaque fois (ou presque), on peut disposer d'une représentation
  intégrale ; ça serait dommage de ne pas en parler.
\end{alert}