% Created 2024-02-10 sam. 22:44
% Intended LaTeX compiler: pdflatex
\documentclass[11pt]{article}
\usepackage[utf8]{inputenc}
\usepackage[T1]{fontenc}
\usepackage{graphicx}
\usepackage{longtable}
\usepackage{wrapfig}
\usepackage{rotating}
\usepackage[normalem]{ulem}
\usepackage{amsmath}
\usepackage{amssymb}
\usepackage{capt-of}
\usepackage{hyperref}
\author{Benjamin Dosse}
\date{2024-02-10 sam.}
\title{Feuille de route}
\hypersetup{
 pdfauthor={Benjamin Dosse},
 pdftitle={Feuille de route},
 pdfkeywords={},
 pdfsubject={},
 pdfcreator={Emacs 28.3 (Org mode 9.5.5)}, 
 pdflang={English}}
\begin{document}

\maketitle
\begin{center}
Aperçu de la structure du document final. 
\end{center}

\setcounter{tocdepth}{2}
\tableofcontents
\section{Chapitre 1 : processus en dimension 1}
\label{sec:org65c0b25}
On définit le mouvement Brownien fractionnaire (\emph{« mBf »}) et le
mouvement Brownien multifractionnaire (\emph{« mBm »}).
\subsection{Mouvement Brownien fractionnaire}
\label{sec:orgc466ec9}
\subsubsection{Définition}
\label{sec:orgc74afd8}
\begin{itemize}
\item Via l'opérateur de covariance
\item Rappeler le théorème de Kolmogorov (+ cas gaussien)
\item Autres définitions plus loin
\end{itemize}
\subsubsection{Propriétés}
\label{sec:org597e829}
Bien que ces propriétés ne soient pas \emph{stricto censu} nécessaires pour
étudier la régularité des trajectoires du mBf, elles sont
intéressantes pour étudier une autre forme de \emph{régularité} (\emph{e.g.}
prévisibilité, prédictibilité).

\begin{itemize}
\item Cas où le mBf est un mB
\item Auto-similaire ; incréments stationnaires ; inversion du temps
\item N'est pas une semi-martingale
\begin{itemize}
\item Pas vu en cours : vu (Gall2013)\footnote{Jean-François LE GALL, \emph{Brownian Motion, Martingales, and
Stochastic Calculus}, Graduate Texts in Mathematics, Berlin :
Springer-Verlag. 2013, p. XIII, 273, ISBN : 978-3-319-31089-3.}, une semi-martingale peut
définir une intégrale stochastique. Comme le rappel
(Nourdin2012)\footnote{Ivan NOURDIN, \emph{Selected Aspects of Fractional Brownian Motion},
Bocconi and Springer Series, Milan : Springer. 2012, p. IX, 122,
ISBN : 978-88-470-2823-4.},
\begin{quote}
    \emph{This explains why integrating with respect to it is an interesting
and non-trivial problem.}

-- Ivan Nourdin, à propos du mBf.
\end{quote}
\end{itemize}
\item N'est pas un processus de Markov
\end{itemize}
\subsubsection{Autres écritures}
\label{sec:orge047a83}
\begin{itemize}
\item Intégrale stochastique
\begin{itemize}
\item Il faut mentionner que la construction sur la droite est similaire
à la construction sur la demi-droite positive.
\end{itemize}
\item Dans le domaine « fréquence » (\emph{i.e.} « transformée de Fourier »)
\begin{itemize}
\item Je suis perdu : (Ayache2019)\footnote{Antoine AYACHE, \emph{Multifractional Stochastic Fields: wavelets
strategies in multifractional frameworks}, World Scientific. 2019,
p. 236, ISBN : 978-98-145-2567-1.\label{org178f8ac}} propose une construction, mais
je ne comprends pas tout à fait son raisonnement. Ailleurs : trop
évasif. Je ne vois pas l'éléphant au milieu de la pièce\ldots{}
\end{itemize}
\end{itemize}
\subsubsection{Régularité}
\label{sec:org6388444}
À chaque fois, il est utile de préciser « de quel côté » se situe
l'exposant de Hurst par rapport à 1/2.
\begin{itemize}
\item Donner la régularité de Hölder locale
\item Donner la régularité de Hölder globale
\item Donner la dimension de Hausdorff
\begin{itemize}
\item Voir (Morters2010)\footnote{Peter MÖRTERS, PERES Yuval, \emph{Brownian Motion}, Cambridge Series
in Statistical and Probabilistic Mathematics, Cambridge : Cambridge
University Press. 2010, p. XII, 403, ISBN : 978-05-117-5048-9.}, chap. 4.
\end{itemize}
\end{itemize}
\subsection{Mouvement Brownien multifractionnaire}
\label{sec:org571b575}
\subsubsection{Définitions}
\label{sec:orgb14ad7b}
\begin{itemize}
\item On \emph{pourrait} suivre la construction de (Peltier1995)\footnote{Romain François PELTIER et LEVY VEHEL Jacques, \emph{Multifractional
Brownian motion: definition and preliminary results}, Rapport de
recherche INRIA n°2645, 1995.}, mais je
trouve que les démonstrations ne sont pas très jolies.
\item Dans (Benassi1998)\footnote{Albert BENASSI, ROUX Daniel, JAFFARD Stéphane, Elliptic gaussian
random processes. \emph{In :} Rev. Mat. Iberoam. 13 (1997), no. 1,
pp. 19–90}, on part de l'expression dans le domaine «
fréquence ».
\item Dans (Ayache2019)\textsuperscript{\ref{org178f8ac}}, l'exposé est le suivant : un générateur
(dans le domaine « fréquence ») est introduit, on étudie quelques
unes de ses propriétés, puis on particularise au mBm. De même, la
décomposition en ondelettes se fera avec un générateur.
\end{itemize}
\subsubsection{« Régularité »}
\label{sec:org649f863}
C'est ici que la structure expose le plus vivement ses faiblesses \emph{si
on suit (Ayache2019)\textsuperscript{\ref{org178f8ac}}} : les propriétés liées à la régularités
sont démontrées dans le cas des champs multifractionnaires\ldots{} On peut
citer les résultats, et les démontrer dans le chapitre suivant ?

\section{Chapitre 2 : passage en dimension quelconque}
\label{sec:orga2e90c2}
\subsection{Généralisations}
\label{sec:org0454e05}
On présente deux généralisations possibles : le champ et le drap.
\begin{itemize}
\item Le \emph{champ} est plus régulier (au sens de l'isotropie : la direction
d'une limite n'influe pas la valeur de la limite) que le
\emph{drap}. Image mnémo. : un lit défait.
\item Champ : norme, drap : produit tensoriel.
\end{itemize}
\subsection{Cas du mBf}
\label{sec:org8ccdf5e}
\subsubsection{Isotrope}
\label{sec:org7930a3f}
\begin{itemize}
\item Définition
\item Écriture alternative
\item Opérateur de covariance
\item Propriétés usuelles
\begin{itemize}
\item Auto-similarité ; incréments
\end{itemize}
\item Version continue
\end{itemize}
\subsubsection{Anisotrope}
\label{sec:orgee8775b}
\begin{itemize}
\item Définition
\item Écriture alternative
\item Opérateur de covariance
\item Propriétés usuelles
\begin{itemize}
\item Auto-similarité ; incréments
\end{itemize}
\item Version continue
\end{itemize}
\subsection{Cas du mBm}
\label{sec:org7f9495c}
\subsubsection{Isotrope}
\label{sec:org4c2f5d5}
\begin{itemize}
\item Définition
\item Écriture alternative
\item Opérateur de covariance
\item Version continue
\end{itemize}
\subsubsection{Anisotrope}
\label{sec:orgf9a607a}
\begin{itemize}
\item Définition
\item Écriture alternative
\item Opérateur de covariance
\item Version continue
\end{itemize}

\section{Chapitre 3 : régularités \& développement en ondelettes}
\label{sec:org1e9c13f}
\subsection{Régularité Hölderienne \& dimension fractale}
\label{sec:orgc9248a4}
Il a été étudié l'existence d'une version continue. Que dire de la
régularité Hölderienne locale et ponctuelle ?

\textbf{N.B. :} demander un avis sur les deux dernières présentations du
cours sur les ondelettes (M1, familles de représentations).

On va suivre (Herbin2002)\footnote{Erick HERBIN, \emph{From N parameter fractional Brownian motions to N
parameter multifractional Brownian motions}, URL :
\texttt{<https://arxiv.org/abs/math/0503182>}.} pour la structure de cette section.
\subsubsection{Cas anisotrope}
\label{sec:orgead4753}
\begin{itemize}
\item Régularité Hölderienne locale
\item Régularité Hölderienne ponctuelle
\item Dimension de Hausdorff
\end{itemize}
\subsubsection{Cas isotrope}
\label{sec:orgd4ff62d}
\begin{itemize}
\item Régularité Hölderienne locale
\item Régularité Hölderienne ponctuelle
\item Dimension de Hausdorff
\end{itemize}

\subsection{Développement en ondelettes}
\label{sec:org531f62d}
Je ne suis absolument pas certain de la position de cette section dans
la structure.

\section{Chapitre X : Cas particulier : mBm avec exposant de Hurst aléatoire}
\label{sec:orgf279e24}
Bon. J'en parle comme d'un « à côté », mais\ldots{} Il n'y a qu'à voir le
traitement de ce sujet dans (Ayache2019)\textsuperscript{\ref{org178f8ac}} pour comprendre que ce
n'est pas lui rendre honneur. Et en fouillant arXiv, c'est plus
frappant encore.
\end{document}