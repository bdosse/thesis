\lecture{2}{ven. 02 déc. 2022 23:40}{Extensions à $\R^d$}
\label{chp:extensions}

On dispose de plusieurs méthodes permettant d'étendre le mouvement
Brownien fractionnaire dans l'espace euclidien $\R^d$ ; on introduit
alors les notions d'extensions isotrope\footnote{On utilise les
  matrices orthogonales pour introduire l'isotropie ; ça revient à
  demander l'invariance par une isométrie~(\cite{ayache2018}).}
(processus de Lévy) et non-isotrope (drap brownien). Ces techniques
sont reprises et étudiées dans le cas d'un mouvement Brownien
multifractionnaire.

\begin{question}
  Ce qui suit évoque des généralisations à $\R^d$ des mBf et mBm ; ne
  devrait-on pas également au moins évoquer le cas du mB standard
  (même si $B = B_H$ avec $H=\frac{1}{2}$) ?
\end{question}

\section{Cas du mBf}
\subsection{Mouvement Brownien fractionnaire de Lévy}
Il s'agit de l'extension isotrope, \emph{i.e.} là où apparaît une
valeur absolue pour le mBf, on fait apparaître une norme sur $\R^d$ ;
dès lors, l'exposant de Hurst est identique quelle que soit la
direction considérée.

\subsection{Drap Brownien fractionnaire}
La norme du mBf de Lévy est remplacée par un produit tensoriel des
covariances unidimensionnelles ; l'exposant de Hurst peut être
différent selon la direction observée.

\section{Cas du mBm}
Les paramètres vivent dans $\R^d$, donc il faut définir une fonction
de Hurst $H:\R^d \to \oset{0}{1}$. Les représentations sont intégrales
: on impose de plus que $H$ soit mesurable.
\subsection{Extension isotrope}
Les constructions sont similaires au mBf.
\subsection{Extension anisotrope}
Les constructions sont similaires au mBf.
\subsection{Régularité des extensions}
La régularité dépend évidemment de la régularité de la fonction de
Hurst $H$ ; \textsc{Herbin}~(\cite{herbin2002}) exhibe une propriété
intéressante (dans le cas isotrope) : si $H \in C^\beta(\R^d)$, alors
on peut contrôler la variance des accroissements en fonction de
$\max{\beta, H(t)}$. On a une propriété similaire dans le cas
anisotrope.

\subsection{Du fBm au mBm}
\vspace{0.2em}
\begin{alert}
  À chaque fois (ou presque), on peut disposer d'une représentation
  intégrale ; ça serait dommage de ne pas en parler.
\end{alert}