\lecture{2}{ven. 02 déc. 2022 23:03}{Processus harmonisables}

La notion de processus harmonisable est présentée par
M. Loève~\cite{loeve1945}. Dans ce chapitre, on notera
$\corr_X(t_1,t_2)$ l'auto-corrélation d'un processus centré $X$ aux
temps $t_1, t_2$, pour autant que $\Esp{|X|^2} < \infty$. S'il existe
une fonction d'auto-corrélation $\gamma$ de variation bornée telle
que\footnote{On observe clairement une composante d'un produit
vectoriel, ou bien un déterminant. Essayer d'investiguer dans la
direction de ces notations.}  \[ \corr_X(t_1,t_2) = \iint e^{i(t_1s_1
- t_2s_2)} d\gamma(s_1,s_2) ,\] alors on dira que $\corr_X$ est
\emph{harmonisbale}.

\nomenclature{$\corr_X$}{Fonction d'auto-corrélation d'un processus
  $X \in L^2(\Omega)$}

On dira qu'un processus $X\in L^2(\Omega)$ est un \emph{processus
  harmonisable} s'il existe une variable aléatoire $Y\in L^2(\Omega)$
telle que $\corr_Y$ est harmonisable et telle que
\[ X(t) = \int e^{its} dY(s) \] où l'égalité est entendue au sens de
la convergence presque sûre. Nous allons montrer qu'un processus est
harmonisable si, et seulement si, sa fonction d'auto-corrélation est
harmonisable. Nous donnerons ensuite des \emph{représentations
  harmonisables} de tels processus, tout en donnant quelques
propriétés remarquables de ceux-ci.


\section{Représentation en série orthogonale}
\section{Représentation intégrale}
\section{Représentation en moyenne mobile}
\section{Le cas du mouvement Brownien (multi)fractionnaire}
