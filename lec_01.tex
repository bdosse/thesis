\lecture{1}{mar. 25 oct. 2022 08:51}{Mouvement Brownien (multi)fractionnaire}

On peut définir le mouvement Brownien (standard) au moyen de sa
représentation intégrale. On peut également le définir au moyen de sa
fonction moyenne $\mu$ et son opérateur de covariance $K$. Plus
précisément, si $B = \{B(t) : t \in \coset{0}{\infty}\}$ désigne un
mouvement Brownien standard sur l'espace de probabilité filtré
$\left(\Omega, \cA, \sF, \nP\right)$, alors la fonction moyenne $\mu$
est donnée par $\mu(t) = 0$ et la fonction de covariance est donnée
par $K(t,s) = \min(t,s)$. Cette définition ne permet pas de préserver
une forme de \emph{\og mémoire \fg{}} du processus ; notion définie
\emph{via} la Définition~\ref{def:memoire-proc}.

Ce chapitre introduit le mouvement Brownien fractionnaire, et établit
ces propriétés mémorielles selon la valeur prise par \emph{l'exposant
de Hurst} $H$. 

\section{Mouvement Brownien fractionnaire}
\subsection{Définition et propriétés}
\subsection{Régularité et auto-similarité}
\subsection{Intégration}

\section{Mouvement Brownien multifractionnaire}
\subsection{Définition et propriétés}
\subsection{Liens avec le mBf}
\subsection{Régularité}
