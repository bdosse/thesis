\lecture{1}{mar. 25 oct. 2022 08:51}{Mouvement Brownien (multi)fractionnaire}
\label{chp:fbm-mbm}

On peut définir le mouvement Brownien (standard) au moyen de sa
représentation intégrale. On peut également le définir au moyen de sa
fonction moyenne $\mu$ et son opérateur de covariance $K$. Plus
précisément, si $B = \{B(t) : t \in \coset{0}{\infty}\}$ désigne un
mouvement Brownien standard sur l'espace de probabilité filtré
$\left(\Omega, \sA, \sF, \nP\right)$, alors la fonction moyenne $\mu$
est donnée par $\mu(t) = 0$ et la fonction de covariance est donnée
par $K(t,s) = \min(t,s)$.

Ce chapitre introduit le mouvement Brownien fractionnaire, où la
régularité est contrôlée par une constante $H$ appelée exposant de
Hurst. Après l'étude de quelques propriétés, notamment en ce qui
concerne la régularité (sous différentes acceptions) des trajectoires,
le mouvement Brownien multifractionnaire est introduit. L'évolution la
plus notable entre ces deux processus est le choix de l'exposant de
Hurst : alors qu'il est constant pour le premier, on considère une
fonction qui dépend du temps pour le second. Une conséquence immédiate
est l'évolution, au long des trajectoires, de la régularité de
celles-ci.

\section{Représentation harmonisable}
La notion de processus harmonisable est présentée par
M. \textsc{Loève}~(\cite{loeve1978}). Dans ce chapitre, on notera
$\corr_X(t_1,t_2)$ l'auto-corrélation d'un processus centré $X$ aux
temps $t_1, t_2$, pour autant que $\Esp{|X|^2} < \infty$. S'il existe
une fonction d'auto-corrélation $\gamma$ de variation bornée telle
que\footnote{On observe clairement une composante d'un produit
  vectoriel, ou bien un déterminant. Essayer d'investiguer dans la
  direction de ces notations.}
\[ \corr_X(t_1,t_2) = \iint e^{i(t_1s_1 - t_2s_2)} d\gamma(s_1,s_2)
  ,\] alors on dira que $\corr_X$ est \emph{harmonisable}.

\begin{prerequis}
  Il faut introduire l'intégrale de Riemann-Stieltjes pour les
  processus (\cite[p.~138]{loeve1978}) pour donner un sens à la double
  intégrale ci-dessus et à l'intégrale ci-dessous.
\end{prerequis}

\nomenclature{$\corr_X$}{Fonction d'auto-corrélation d'un processus
  $X \in L^2(\Omega)$}

On dira qu'un processus $X\in L^2(\Omega)$ est un \emph{processus
  harmonisable} s'il existe une variable aléatoire $Y\in L^2(\Omega)$
telle que $\corr_Y$ est harmonisable et telle que
\begin{equation}
  \label{eqn:rep-harm-proc}
  X(t) = \int e^{its} dY(s)
\end{equation}
où l'égalité est entendue au sens de la convergence presque sûre. Nous
allons montrer qu'un processus est harmonisable si, et seulement si,
sa fonction d'auto-corrélation est harmonisable. Nous donnerons
ensuite des \emph{représentations harmonisables} de tels processus,
tout en donnant quelques propriétés remarquables de ceux-ci.

\begin{alert}
  On obtient une \og transformée de \emph{Fourier-Stieltjes} \fg{}. On
  peut trouver une sorte d'inverse (convergence en moyenne
  quadratique).
\end{alert}

\subsection{Représentation en série orthogonale}

\begin{alert}
  Ajout d'un post-it (\cite[p.~143]{loeve1978}).
\end{alert}

\textsc{Loève} présente deux décompositions : un développement à la
Taylor et une décomposition par des fonctions dites
\emph{propres}\footnote{Utilisation du théorème de Mercer, à renvoyer
  en annexe ?}. Après un développement peu amusant, on définit ce
qu'est un processus de second ordre stationnaire. En particulier un
processus de second ordre\footnote{C'est à dire, qui admet des moments
  d'ordre deux finis} qui est de plus stationnaire est un processus de
second ordre stationnaire.

\subsection{Représentation intégrale}
On donne ici une condition nécessaire et suffisante pour obtenir une
représentation de la forme~(\ref{eqn:rep-harm-proc}) : être
stationnaire et continu en un point.

\begin{alert}
  Les liens ne sont pas très clairs entre les différentes
  décompositions, ni comment elles peuvent être appliquées afin
  d'extraire de l'information sur les processus.
\end{alert}

\section{Mouvement Brownien fractionnaire}
\subsection{Définition et propriétés}

Le mouvement Brownien fractionnaire (on écrira mBf à partir de
maintenant) est \emph{une} gé\-né\-ra\-li\-sa\-tion du mouvement
Brownien. Cette gé\-né\-ra\-li\-sa\-tion permet de définir la
régularité (au sens de Hölder) des trajectoires du processus. Pour
cela, on utilise l'exposant (dit \emph{de Hurst}) $H$.

\begin{question}
  Problème de l'estimation de $H$ : a-t-il sa place dans ce document ?
\end{question}

Lévy propose d'utiliser l'intégrale de Riemann-Liouville(-Itô) pour
définir le mBf :
\begin{equation}
  B_H = \frac{1}{\Gamma(H + \frac{1}{2})} \int_0^t (t-s)^{H-\frac{1}{2}} dB(s)
\end{equation}

Or, d'après~\cite{mandelbrot1968}, \emph{\og cette intégrale accorde
  bien trop d'importance à l'origine pour beaucoup d'applications
  \fg{}}. La re\-pré\-sen\-ta\-tion de Weyl est alors proposée :
\begin{gather}
  B_H(0) = 0\\
  B_H(t) = \frac{1}{\Gamma(H+\frac{1}{2})}\left(\int_{-\infty}^0(t-s)^{H-\frac{1}{2}}-(-s)^{H-\frac{1}{2}}dB(s) + \int_0^t(t-s)^{H-\frac{1}{2}}dB(s) \right)
\end{gather}
où les égalités ont lieu p.s.

\begin{alert}
  Donner la définition du mBf \emph{via} l'espérance.
\end{alert}

\begin{alert}
  On obtient gratuitement une représentation intégrale du mBf grâce à
  la re\-pré\-sen\-ta\-tion de Weyl. Il suffira de calculer l'espérance.
\end{alert}

\subsection{Propriétés et régularité}
L'exposant de Hurst modifie la régularité trajectorielle du fBm, mais
modifie également quelques propriétés en comparaison du mouvement
Brownien standard. Le mBf est auto-similaire, mais ses incréments ne
sont pas indépendants, et il ne s'agit pas d'une semimartingale.

\begin{prerequis}
  On peut définir ce qu'est une semimartingale, mais il faut alors
  définir ce qu'est une martingale locale. Soit on suppose que c'est
  connu, soit on donne tout de même quelques propriétés de ces objets.
\end{prerequis}

On peut définir \emph{plusieurs} intégrales stochastiques pour le fBm
: voir~\cite[Chp.~II-V]{biagini2008}. C'est moins agréables qu'avec le
mouvement Brownien standard.

Tout comme pour le mouvement Brownien standard, on utilise le théorème
de Kolmogorov-Centsov pour établir la régularité des trajectoires du
fBm.

\begin{alert}
  \textsc{Biagini} énonce que le fBm possède une mémoire longue. Cette
  propriété est davantage étudiée pour les séries temporelles.
\end{alert}

\section{Mouvement Brownien multifractionnaire}
Un inconvénient considérable du fBm est son incapacité à faire varier
la régularité des trajectoires au cours du temps. Le mouvement
Brownien multifractionnaire (noté mBm dans la suite) permet de se
passer de cette limitation en utilisant une fonction exposant de Hurst
$H(t)$.

\subsection{Définition et propriétés}
\textsc{Ayache}~(\cite{ayache2018}) introduit le mBm au moyen d'une
représentation intégrale et de générateurs.

\begin{prerequis}
  Inclure le passage sur la transformation de Fourier : elle est utile
  pour quelques articles.
\end{prerequis}

Il existe des liens évidents avec le mBf : il suffit de choisir une
fonction constante.

\begin{question}
  Que peut-on dire de l'auto-similarité du mBm ?
\end{question}

\begin{prerequis}
  On peut définir (au moins) une intégrale stochastique pour le
  mBm. C'est assez difficile, dans le sens où il faut utiliser des
  outils qui (je crois) n'ont pas encore été vu ou ne sont pas au
  programme actuellement.
\end{prerequis}

\subsection{Régularité}
Sont données quelques propriétés trajectorielles, dont des conditions
nécessaires et suffisantes sur la fonction $H$ pour la continuité du
mBm. Des résultats concernant la régularité (globale, locale, et
ponctuelle) de Hölder sont démontrés.