%\lecture{1}{dim. 16 oct. 2022 20:44}{Construction initiale}

\section{Intégrale d'Itô}

Quelques rappels : définition comme limite, la règle de la chaîne,
quelques propriétés remarquables.

\section{Intégrale de Stratonovich}

Définitions, comparaison avec l'intégrale d'Itô. Beaucoup de propriété
de l'intégrale d'Itô sont retrouvées ici. Équivalence dans la
formulation de certaines EDS. Signaler que ces formulations ne sont
pas toujours équivalentes, et donner un contre-exemple.

\section{EDS et analyse numérique}

Pas vraiment intéressant. On donnera quelques définitions, propriétés
(que l'on démontre dans le cas stochastique), et on essaie de trouver
des EDS qui diverge rapidement pour Euler-Maruyama.

\subsection{Les algorithmes}

Pour le moment, Euler-Maruyama et Milstein. Il faudra introduire la
notion de développement de Taylor-Itô. Les démonstrations reposeront
sur ce développement (enfin, surtout pour donner l'intuition
nécessaire).

On peut évoquer les méthodes de Runge-Kutta stochastiques, mais ça
demande beaucoup plus de développement que dans le cas des équations
différentielles ordinaires.

\section{Champs gaussiens}

On fait intervenir l'article de \cite{bierme} : les champs
gaussiens. On utilise alors un développement en ondelettes, ce qui
permet de traiter le signal aléatoire d'une façon plus lisse (on sait
que les processus de Wiener possède quelques propriétés liées à leur
régularité, mais qu'en est-il lorsque l'on applique un produit
tensoriel ?)

