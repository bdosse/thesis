\lecture{3}{ven. 02 déc. 2022 23:40}{Extensions du mBf}

Dans ce chapitre, on introduit la notion de champs gaussien, et leurs
propriétés relatives.

On dispose de plusieurs méthodes permettant d'étendre le mouvement
Brownien fractionnaire dans l'espace euclidien $\R^n$ ; on introduit
alors les notions d'extensions isotrope\footnote{On utilise la notion
de matrice orthogonale pour introduire l'isotropie ; ça revient à
demander l'invariance par une isométrie.} (processus de Levy) et
non-isotrope (drap brownien). Ces techniques seront reprises et
étudiées dans le cas d'un mouvement Brownien multifractionnaire.

\section{Mouvement Brownien fractionnaire de Levy}
\subsection{Régularité}
\section{Drap Brownien fractionnaire}
\subsection{Régularité}