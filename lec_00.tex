\lecture{0}{mer. 5 avril 2023 15:03}{Introduction, Préambule, etc.}

\chapter*{Remerciements}
\markboth{\textsc{Remerciements}}{}
\chapter*{Introduction}
\markboth{\textsc{Introduction}}{}


\section{À faire / À chercher}
\begin{itemize}
\item Introduction des différents chapitres
\item Liens avec les séries temporelles
  \begin{itemize}
  \item On étudie ces liens \emph{après} avoir exposé la théorie sur
    le mBm.
  \item En particulier, on veut prédire le comportement d'une série
    temporelle. On peut estimer sa fonction exposant de Hurst, mais
    A. Deliège, dans son mémoire, semble indiquer que ce n'est pas une
    façon de faire commune.
  \item Pour estimer l'exposant de Hölder ponctuel d'une série
    temporelle suivant le modèle du mBm, voir par
    exemple~\cite{jin2017}. Pour l'estimation de l'exposant de Hurst,
    on peut se référer à~\cite{coeurjolly2005, lebovits2018}
  \item Les références~\cite{garcin2016, garcin2021} peuvent être
    utiles à cet égard.
  \end{itemize}
\item Concernant la représentation en ondelettes, prendre garde à
  certaines subtilités (\cite{ayache2010}).
\item Temps local d'un processus : \st{applicable aux semimartingales,
    et le mBf n'est pas une semimartingale (ni le mBm).}
  $\longrightarrow$ Ayache définit autrement le temps local,
  voir~\cite[Sec.~2.2]{ayache2018}. On pourra faire la distinction
  avec les semimartingales, c'est un cas plus facile à traiter
  (voir~\cite[p.~13-14]{yen2013}), et s'aider
  de~\cite[Sec.~10.1]{biagini2008} (pour le fBm).
\item Trouver des applications : le mBf est populaire, le mBm un peu
  moins (voir
  \href{https://team.inria.fr/anja/english-theoretical-aspects/multifractional-brownian-motion/multifractional-brownian-motion-bibliography/}{cette page de l'INRIA}
  au cas où)
\item Ajouter quelques illustrations (utiliser \textsc{R} +
  \textsc{Python} pour ne pas perdre trop de temps sur ce point).
\end{itemize}