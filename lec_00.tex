\lecture{0}{mer. 5 avril 2023 15:03}{Introduction, Préambule, etc.}

\chapter*{Remerciements}
\markboth{\textsc{Remerciements}}{}
\chapter*{Introduction}
\markboth{\textsc{Introduction}}{}


\section{À faire / À chercher}
\begin{itemize}
\item Revoir l'introduction
\item Liens avec les séries temporelles
\item Temps local d'un processus : \st{applicable aux semimartingales,
    et le mBf n'est pas une semimartingale (ni le mBm).}
  $\longrightarrow$ Ayache définit autrement le temps local,
  voir~\cite[Sec.~2.2]{ayache2018}. On pourra faire la distinction
  avec les semimartingales, c'est un cas plus facile à traiter
  (voir~\cite[p.~13-14]{yen2013}), et s'aider
  de~\cite[Sec.~10.1]{biagini2008} (pour le fBm).
\item Trouver des applications : le mBf est populaire, le mBm un peu
  moins (voir
  \href{https://team.inria.fr/anja/english-theoretical-aspects/multifractional-brownian-motion/multifractional-brownian-motion-bibliography/}{cette page de l'INRIA}
  au cas où)
\item Ajouter quelques illustrations (utiliser \textsc{R} +
  \textsc{Python} pour ne pas perdre trop de temps sur ce point).
\end{itemize}